\documentclass[index]{subfiles}

\begin{document}

% See https://www.overleaf.com/learn/latex/How_TeX_macros_actually_work%3A_Part_4
\def\columntwo#1#2{
    \begin{paracol}{2}
        \begin{center}Pros\end{center}
        \begin{itemize}
            #1
        \end{itemize}

        \switchcolumn

        \begin{center}Cons\end{center}
        \begin{itemize}
            #2
        \end{itemize}

    \end{paracol}
}

\subsection*{Harvard Step Test}

\columntwo{
    \item There is little amount of equipment needed
    \item Sub-maximal test, which doesn't require maxinum strength, so novices can participate
    \item Measures physiological condition (heart rate), which is less subject to things like motivation and pacing
}{
    \item Doesn't measure true strength, but rather pulse, and uses it to predict aerobic capacity
    \item Doesn't take into consideration individual differences in hearts
    \item Innacurate measuring of heart rate may cause huge errors
    \item Could become a maximal test if the individual does not move around much and cannot step onto the bench
}

\subsection*{Stork Stand}

\columntwo {
    \item There is no equipment needed
    \item Instructions are simple and can be conducted anywhere
}{
    \item Person needed to administer the test
    \item The test may not be reliable if the individual has practice balancing on one leg
    \item The reliability of the test is questionable because environment could have an impact
}

\subsection*{Maximum Push-ups}

\columntwo{
    \item There is no equipment needed
    \item Instructions are simple and can be conducted anywhere
    \item Maximal test, meaning that it's not a prediction
}{
    \item It's a maximal test, so motivation and pacing may affect the results, subject may be injured
    \item Not done in a lab, so the environment, such as the material and rigidity of the floor, may affect results
}

\subsection*{Sit and Reach}

\columntwo{
    \item The instructions are simple and the test may be conducted any where
}{
    \item The subject may not be able to reach the box, or go over the length of the box
    \item Requires a marked box
    \item Biological differences are not accounted for, the individual may have disproportionately large arms compared to legs, or vice-versa.
    \item Subject may be affected by motivation to push through pain, or get injured
}

\subsection*{Illinois Agility Test}

\columntwo{
    \item Little equipment needed, subject does not need a huge amount of training to complete the test
}{
    \item The test may be unreliable if the subject gains experience with the test
    \item The validity of the test is questionable because the individual's speed has a large impact on agility
    \item Is a maximal test, so subject may be affected by motivation and pacing
    \item It takes a long time to itemize things
    \item May become a test of strength if subject is unable to get off the ground
    \item The environment of the test, such as material of shoes, weather, or material of track, may have an impact on the reliability of the test
}

\subsection*{35-meter Dash}

\columntwo{
    \item There is little equipment needed, subject does not need a huge amount of training to complete the test
    \item The instructions are simple
}{
    \item Maximal test, so subject may be affected by motivation and pacing
    \item The validity of the test is questionable, because in a 35-second sprint reaction time plays a huge role
    \item The reliability of the test is also questionable, because the subject may be more experienced in take off and acceleration.
    \item The environment of the test, such as material of shoes, weather, or material of track, may have an impact on the reliability of the test
}

\subsection*{Ruler Drop}

\columntwo{
    \item There is little equipment needed, subject does not need a huge amount of training to complete the test
    \item The instructions are simple
}{
    \item The reliability of the test is questionable, because the subject may become more accustomed to the ruler after a while. The subject could also get lucky by preemtively closing their hand. The angle of the ruler and height of drop may also vary
    \item Requires another person to administer the test
}

\subsection*{Ball toss}

\columntwo{
    \item Little equipment needed, subject does not need a huge amount of training to complete the test
    \item Instructions are simple
    \item Sub-maximal test, number of ball tosses is a prediction of what coordination is
}{
    \item The environment, such as the material of the wall
    \item The reliability of the test is \textbf{extremely} questionable. First, the ball tossing skill of the individual could improve over time, launching the ball at an optimal height or angle could be hugely beneficial.
    \item The validity of the test is questionable, because if the subject misses catching the ball it becomes a test of speed to chase after the ball, or if the subject is able to throw the ball faster, it becomes a test of strength.
    \item There is no way to record half a ball caught, say when the time limit is about to be reached
}

\subsection*{Vertical Jump (Sergeant Jump Test)}


\columntwo{
    \item Required equipment readily available
    \item Little training required to perform the test
    \item The instructions are simple
}{
    \item Maximal test, so motivation may affect the amount of distance jumped
    \item The reliability of the test is questionable since over time the subject may become more accustomed to standing jump techniques
    \item Subject may be injured from maximal jumping tests
}

\end{document}