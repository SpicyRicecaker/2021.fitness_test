% Font size, paper type
\documentclass[12pt]{article}
% Aesthetic margins
\usepackage[margin=1in]{geometry}
% Core math packages,
% Mathtools loads amsmath, and amsmath gives basic math symbs
% Amsfonts & amssymb are misc. symbols you might need
\usepackage{mathtools, amsfonts, amssymb}
% Links in a pdf
\usepackage{hyperref}
% Use in pictures, graphs, and figures
\usepackage{graphicx}
% Header package
\usepackage{fancyhdr}
% Underlining with line breaks
\usepackage{ulem}
% Adjust accordingly given warning messages
\setlength{\headheight}{15pt}
% So we can more easily format text with pictures
\usepackage{float}
% Images and drawing graphs
\usepackage{tikz}
% Something about bits and stuff
\usepackage[T1]{fontenc}
% Set the incoding to unicode instead of ascii
\usepackage[utf8]{inputenc}
% Set the font to arial
% \usepackage{fontspec}
% \usepackage{mathspec}
% \setmainfont{Arial}
% \setmathrm{Arial}
% \setmathfont(Digits,Latin){Arial}
% Defines stuff about tables
\usepackage{tabularx}
% Similar to `lorem*[n]` in emmet
\usepackage{lipsum}
% Spits columns into two, the sane way
\usepackage{paracol}
% Layout of two columns
% \usepackage{multicol}
% Removes automatic indent at the start of a paragraph
% \usepackage{indentfirst} 
% Modularity! Includes other .tex files, precompiled
% Use \subfile{path_of_doc} to include other filesr
\usepackage{subfiles}

% Sets footer
\pagestyle{fancy}
% Removes default footer style
\fancyhf{}

\rhead{
\thepage{}
}

% Makes links look more appealing
\hypersetup{
colorlinks=true,
linkcolor=blue,
filecolor=magenta,      
urlcolor=cyan,
}

\begin{document}
\title{Fitness Test Evaluation Experiment}
\author{by Shengdong Li}
\date{21 October 2021}
\maketitle

\section*{Evaluation of Various Tests of Fitness}

\subfile{proscons}

\section*{The Importance of the Physical Readiness Questionnaire (PAR-Q)}

The physical readiness questionnaire is a way to validate that a participant is fit for physical activity. It asks questions about whether the participant has any conditions, is taking medication, or feels anything odd when they exercise. If so, then the participant should not be allowed to participate, because otherwise, it could endanger their health. Therefore the physical readiness questionnaire is important because it screens to prevent human injury or loss of life.

\section*{Health-Related Fitness vs. Performance-Related Fitness}

Performance-related fitness refers to an individual's ability to perform in a specific sport, like soccer, basketball, or swimming. It's different from health-related fitness, which refers more to the ability to accomplish everyday tasks, like standing up and walking, and the importance of exercise in maintaining health. In this experiment, we used sub-maximal and maximal tests to measure various metrics, which means we were measuring performance-related fitness.

\section*{Sub-Maximal Tests vs. Maximal Tests of Human Performance}

Sub-maximal tests measure an individual's performance at a level below their maximum capability. The resulting measured values are used in conjunction with previous data to predict the individual's maximum capability. This has various advantages and disadvantages compared to maximal tests, which measure an individual's performance at their maximum capability. The advantages are that sub-maximal tests are often simpler, novices can participate with less instruction, and less risk of injury because they aren't pushing their bodies to the limits. The fact that values are used to predict maximum strength, means that they are less subject to physiological factors such as motivation, and skill and pacing play less of a role in skewing results. The disadvantage is that values aren't a true representation of individual strength, and generally sub-maximal tests are less specific and valid than maximal tests.

\section*{Why is fitness testing important?}

Fitness testing is incredibly important both for scientists, coaches, and their athletes, because it prevents stagnation, allows for scientific studies, and presents a variety of health benefits for athletes.

First of all, fitness testing provides the necessary data required for athletes to evaluate their performance. "Careful monitoring" from scientists and supplemental metrics from smaller competitions can provide "objective assessments" of an athlete's performance over time. This data may to used to identify "additional strategies", as well as perfect and fine-tune current ones, which makes it very important for athletes as a regular part of their training regime because it allows them to min-max and improves performance.

Fitness testing is also useful for actual scientists because it allows them to learn more about human limits and the efficacy of certain medications. Scientists have been "fascinated" with the "amazing performance capabilities" of modern athletes. By testing athletes, scientists naturally get information about said athletes. Scientists may also find the "effectiveness" of "approved medications", and make adjustments and find side effects of the medication. This is a mutually beneficial relationship between scientists and athletes, where athletes can get better medication and performance results while scientists make discoveries and better drugs.

Finally, fitness testing allows athletes to stay healthy over the long run. Scientists may screen athletes for early symptoms of diseases and sports injuries, such as "likelihood of overtraining" and "trends toward anemia", which may help athletes avoid suffering and keep on working for longer. They may also make "nutritional evaluation[s]" that may help with the athlete directly in terms of their physical performance.

Overall, fitness testing is mutually beneficial for both scientists, in making discoveries, and athletes, in making improvements, as well as staying healthy, and is a very important thing often overlooked by others on their journey of self-improvement. We should spread awareness of fitness testing and how and why athletes should take these tests, as well as make them more accessible for normal athletes.

\subsection*{Works Cited}
Martin \& Coe (1997), Scientific Evaluation of Health and Fitness
\end{document}
