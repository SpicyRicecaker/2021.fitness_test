% Font size, paper type
\documentclass[12pt]{article}
% Aesthetic margins
\usepackage[margin=1in]{geometry}
% Core math packages,
% Mathtools loads amsmath, and amsmath gives basic math symbs
% Amsfonts & amssymb are misc. symbols you might need
\usepackage{mathtools, amsfonts, amssymb}
% Links in a pdf
\usepackage{hyperref}
% Use in pictures, graphs, and figures
\usepackage{graphicx}
% Header package
\usepackage{fancyhdr}
% Underlining with line breaks
\usepackage{ulem}
% Adjust accordingly given warning messages
\setlength{\headheight}{15pt}
% So we can more easily format text with pictures
\usepackage{float}
% Images and drawing graphs
\usepackage{tikz}
% Something about bits and stuff
\usepackage[T1]{fontenc}
% Set the incoding to unicode instead of ascii
\usepackage[utf8]{inputenc}
% Set the font to arial
% \usepackage{fontspec}
% \usepackage{mathspec}
% \setmainfont{Arial}
% \setmathrm{Arial}
% \setmathfont(Digits,Latin){Arial}
\usepackage{tabularx}

% Sets footer
\pagestyle{fancy}
% Removes default footer style
\fancyhf{}

\rhead{
\thepage{}
}

% Makes links look more appealing
\hypersetup{
colorlinks=true,
linkcolor=blue,
filecolor=magenta,      
urlcolor=cyan,
}

% \usepackage{multicol}
% \usepackage{indentfirst}

\begin{document}
\title{Fitness Test Evaluation Experiment}
\author{by Shengdong Li}
\date{21 October 2021}
\maketitle

\section*{Evaluation of Various Tests of Fitness}

\subsection*{Harvard Step Test}

\begin{minipage}[t]{0.5\textwidth}
    \begin{center}Pros\end{center}
    \begin{itemize}
        \item There is little amount of equipment needed
        \item Sub-maximal test, which doesn't require maxinum strength, so novices can participate
        \item Measures physiological condition (heart rate), which is less subject to things like motivation and pacing
    \end{itemize}
\end{minipage}
\begin{minipage}[t]{0.5\textwidth}
    \begin{center}Cons\end{center}
    \begin{itemize}
        \item Doesn't measure true strength, but rather pulse, and uses it to predict aerobic capacity
        \item Doesn't take into consideration individual differences in hearts
        \item Innacurate measuring of heart rate may cause huge errors
        \item Could become a maximal test if the individual does not move around much and cannot step onto the bench
    \end{itemize}
\end{minipage}

\subsection*{Stork Stand}

\begin{minipage}[t]{0.5\textwidth}
    \begin{center}Pros\end{center}
    \begin{itemize}
        \item There is no equipment needed
        \item Instructions are simple and can be conducted anywhere
    \end{itemize}
\end{minipage}
\begin{minipage}[t]{0.5\textwidth}
    \begin{center}Cons\end{center}
    \begin{itemize}
        \item Person needed to administer the test
        \item The test may not be reliable if the individual has practice balancing on one leg
    \end{itemize}
\end{minipage}

\subsection*{Maximum Push-ups}

\begin{minipage}[t]{0.5\textwidth}
    \begin{center}Pros\end{center}
    \begin{itemize}
        \item There is no equipment needed
        \item Instructions are simple and can be conducted anywhere
        \item Maximal test, meaning that it's not a prediction
    \end{itemize}
\end{minipage}
\begin{minipage}[t]{0.5\textwidth}
    \begin{center}Cons\end{center}
    \begin{itemize}
        \item It's a maximal test, so motivation and pacing may affect the results
        \item Not done in a lab, so the environment, such as the material and rigidity of the floor, may affect results
    \end{itemize}
\end{minipage}

\subsection*{Sit and Reach}

\begin{minipage}[t]{0.5\textwidth}
    \begin{center}Pros\end{center}
    \begin{itemize}
        \item The instructions are simple and the test may be conducted any where
    \end{itemize}
\end{minipage}
\begin{minipage}[t]{0.5\textwidth}
    \begin{center}Cons\end{center}
    \begin{itemize}
        \item The subject may not be able to reach the box, or go over the length of the box
        \item Requires a marked box
        \item Biological differences are not accounted for, the individual may have disproportionately large arms compared to legs, or vice-versa.
        \item Subject may be affected by motivation to push through pain, or get injured
    \end{itemize}
\end{minipage}

\subsection*{Illinois Agility Test}

\begin{minipage}[t]{0.5\textwidth}
    \begin{center}Pros\end{center}
    \begin{itemize}
        \item The instructions are simple and the test may be conducted any where
    \end{itemize}
\end{minipage}
\begin{minipage}[t]{0.5\textwidth}
    \begin{center}Cons\end{center}
    \begin{itemize}
        \item The subject may not be able to reach the box, or go over the length of the box
        \item Requires a marked box
        \item Biological differences are not accounted for, the individual may have disproportionately large arms compared to legs, or vice-versa.
        \item Subject may be affected by motivation to push through pain, or get injured
    \end{itemize}
\end{minipage}

\end{document}